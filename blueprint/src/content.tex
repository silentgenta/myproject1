% In this file you should put the actual content of the blueprint.
% It will be used both by the web and the print version.
% It should *not* include the \begin{document}
%
% If you want to split the blueprint content into several files then
% the current file can be a simple sequence of \input. Otherwise It
% can start with a \section or \chapter for instance.
\begin{example}
  Example 2.1. The set B ={0, 1}is a semiring under the usual addition and multipli-cation, with the notable exception that 1 +1 =1. This is usually called theboolean semiring; see [22, pg. 12].
https://github.com/leanprover-community/mathlib4/blob/7deb334c5f5104f4edad1a6396dd02a8cddefb86/Mathlib/Algebra/Order/Ring/Idempotent.leanL66-L87
\end{example}


semiring Risadditively idempotentif
https://github.com/leanprover-community/mathlib4/blob/7deb334c5f5104f4edad1a6396dd02a8cddefb86/Mathlib/Algebra/Order/Kleene.leanL101-L117



the tropical semiring
https://github.com/leanprover-community/mathlib4/blob/7deb334c5f5104f4edad1a6396dd02a8cddefb86/Mathlib/Algebra/Tropical/Basic.leanL481-L487



Definition 2.4. Acongruenceon a semiring Ris a subset E⊆R×Rwith the following properties:
https://github.com/leanprover-community/mathlib4/blob/7deb334c5f5104f4edad1a6396dd02a8cddefb86/Mathlib/Algebra/RingQuot.leanL75-L110



Definition 2.5. Let Rbe a semiring. Thetwisted producton R×Ris defined by
(a, b)  (c, d) :=(ac + bd, ad + bc)
for a, b, c, d ∈R.
たぶんないので作らないといけない



Lemma 2.6. Suppose Ris a commutative semiring. A subset E⊆R×Ris a congruence on Rif and only if Esatisfies conditions (E1)–(E3), (I1), and the following condition:
(I2)’If (a, b) ∈E, then (c, d) (a, b) ∈Efor every c, d ∈R.
無い
